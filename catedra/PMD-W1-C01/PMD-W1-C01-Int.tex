\documentclass{beamer}
\usetheme{metropolis}           % Use metropolis theme
\usepackage{outlines}
\usepackage{listings}
\usepackage{pdfpages}
%\usepackage{minted}
%\usepackage{pygmentiz}


\title{Procesamiento Masivo de Datos}
\date{\today}
\author{Alex Di Genova}
\institute{Universidad de O'higgins}
\begin{document}
  \maketitle
  
  
  
  \begin{frame}{Outline}
  \tableofcontents
\end{frame}

\section{Bienvenida curso de PMD}
  \begin{frame}{Presentación}     
   % \begin{itemize}
\only<1>{
  Alex Di Genova
    \begin{outline}
    \1 2003--2008 Ingeniero en Bioinformática. 
    \1 2013-2017 Doctor en Sistemas Complejos.
    \1 2017-2021 Postdoctorado en algoritmos y cáncer (Francia).
    \1 2022 - Profesor Asistente UOH.
    \2 Di Genoma Lab 
    	\3 Combinamos el desarrollo de nuevos algoritmos, análisis de genomas y tecnologías ómicas de última generación para estudiar sistemas biológicos complejos.
    \end{outline}
   }
   
  \end{frame}
 
  {
\setbeamercolor{background canvas}{bg=}
 \includepdf[pages=-]{img/wengan.pdf}
}
 
  \section{Planificación curso PMD}
 \begin{frame}{Unidades}
  \only<1>{
    \begin{itemize}
    	\item 4 Unidades (14 semanas)
		\begin{itemize}
			\item Distribución y Paralelismo (4 semanas)
			\item Modelamiento de Procesamiento Distribuido (4 semanas)
			\item Modelos de Almacenamiento Escalable (3 semanas) 
			\item Bases de datos Distribuidas (3 semanas)
		\end{itemize}
	  \end{itemize}
}
 


  \end{frame}
  
  \begin{frame}{Evaluaciones}
  \begin{outline}
  	
  	\1 Controles (75\%, potenciales fechas de controles):
	\2 Control 1:  Semana del 2 Octubre.
	\2 Control 2:  Semana del 30 Octubre.
	\2 Control 3:  Semana del 27 Noviembre.

	\uncover<2->{
	\1 Tareas (25\%, potenciales fechas de entrega):
	\2 Tarea 1: Semana del 25 Septiembre.
	\2 Tarea 2: Semana del 20 Noviembre. 
	}
	\uncover<3->{
	\1 Examen (40\%, potencial fecha del examen):
	\2 Examen : Semana del 11 Diciembre.
	}
	
  \end{outline}
  \end{frame}
  
  \begin{frame}{Condiciones y Políticas de Evaluación}
  \small
  \begin{outline}
   \1 El promedio de actividades complementarias se considerará como un cuarto control (control IV) y tendrá una ponderación de 25\%. El promedio de controles I,II, III, IV con sus respectivas ponderaciones corresponderán a la nota de cátedra del curso. 
   \1  Examen:  Los alumnos que cuenten con una nota de cátedra menor a 5. Deberán rendir el examen.
   \1 Nota final: Promedio ponderado del examen (40\%) y promedio de los controles (60\%). 
    \1 Los alumnos eximidos recibirán como nota de examen el promedio de las notas de los controles de cátedra. Si lo desean, podrán rendir el examen, en cuyo caso se considerará la nota obtenida sólo si ésta es superior al promedio de las notas de los controles.
  \end{outline}
  \end{frame}
  
  \begin{frame}{Condiciones y Políticas de Evaluación}
  \small
  \begin{outline}      
   \1 Estudiantes que se ausenten a un control tendrán la oportunidad de recuperarlo con el examen. La nota del examen reemplazará la nota más baja de los controles de la asignatura, solo en caso de ser la nota de examen superior.
     \1 Un/a estudiante que cometa plagio sobtendrá un 1,0 en la evaluación y el caso será informado a Escuela de Ingeniería.
  \end{outline}
  \end{frame}

    
  \begin{frame}{Materiales}
  \begin{outline}
  \1 Repositorio GitHub -- \url{https://github.com/adigenova/uohpmd}
  \2 Slides Clases (Lunes y Miercoles : 12:00 - 13:30)
  \2 Código 
   \1 Ayudante : Iván Bozo Catalán  (Jueves 16:15 - 17: 45)
  \uncover<2->{ \1 Ucampus -- \url{https://ucampus.uoh.cl/uoh/2023/2/COM4002}
  \2 Comunicación (Consultas, noticias, evaluaciones)
  \2 Planificación }
 \uncover<3->{  \1 Bibliografía 
  \2 S. Tanenbaum, M. Van Steen. Distributed Systems: Principles and Paradigms (2nd Edition). Prentice Hall, 2006
  \2 P. J. Sadalage, M. Fowler. NoSQL Distilled: A Brief Guide to the Emerging World of Polyglot Persistence. Addison-Wesley Professional, 2012 }
  \end{outline}
  \end{frame} 
  
  \begin{frame}{Resultados de Aprendizaje}
  	\begin{outline}
		\1 Conocer los principios fundamentales de diseño de sistemas distribuidos y paralelos
		 \uncover<2->{ \1 Implementar comunicación entre procesos de una máquina (OpenMP, pthread)}
		 \uncover<3->{ \1 Implementar comunicación entre máquinas (MPI)}
		 \uncover<4->{ \1 Utilizar Nextflow/Hadoop para distribuir tareas computacionales en un clúster. }
		 \uncover<5->{ \1 Conocer la taxonomía de modelos de datos NoSQL, sus lenguajes de consulta y manipulación de datos con modelos NO-SQL.}
		 \uncover<6->{ \1 Construir  y manipular una base de datos distribuida.}
	\end{outline}
  \end{frame}
  

  
  
    \begin{frame}{Consultas?}
    \centering 
    Consultas o comentarios?\\
    Muchas Gracias.
    
    \end{frame}

  
\end{document}