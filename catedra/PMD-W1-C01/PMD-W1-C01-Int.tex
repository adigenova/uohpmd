\documentclass{beamer}
\usetheme{metropolis}           % Use metropolis theme
\usepackage{outlines}
\usepackage{listings}
\usepackage{pdfpages}
%\usepackage{minted}
%\usepackage{pygmentiz}


\title{Procesamiento Masivo de Datos}
\date{\today}
\author{Alex Di Genova}
\institute{Universidad de O'higgins}
\begin{document}
  \maketitle
  
  
  
  \begin{frame}{Outline}
  \tableofcontents
\end{frame}

\section{Bienvenida curso de PMD}
  \begin{frame}{Presentación}     
   % \begin{itemize}
\only<1>{
  Alex Di Genova
    \begin{outline}
    \1 2003--2008 Ingeniero en Bioinformática. 
    \1 2013-2017 Doctor en Sistemas Complejos.
    \1 2017-2021 Postdoctorado en algoritmos y cáncer (Francia).
    \1 2022-2023 Profesor Asistente UOH.
    \1 2023-Presente Profesor Asociado UOH.
    \2 Di Genoma Lab 
    	\3 Combinamos el desarrollo de nuevos algoritmos, análisis de genomas y tecnologías ómicas de última generación para estudiar sistemas biológicos complejos.
    \end{outline}
   }
   
  \end{frame}
 
  {
\setbeamercolor{background canvas}{bg=}
 \includepdf[pages=-]{img/wengan.pdf}
}
 
  \section{Planificación curso PMD}
 \begin{frame}{Unidades}
  \only<1>{
    \begin{itemize}
    	\item 4 Unidades (15 semanas)
		\begin{itemize}
			\item Distribución y Paralelismo (4 semanas)
			\item Modelamiento de Procesamiento Distribuido (5 semanas)
			\item Modelos de Almacenamiento Escalable (3 semanas) 
			\item Bases de datos Distribuidas (3 semanas)
		\end{itemize}
	  \end{itemize}
}
 


  \end{frame}
  
  \begin{frame}{Evaluaciones}
  \begin{outline}
  	
  	\1 Controles (75\%, potenciales fechas de controles):
	\2 Control 1:  Semana del 23 Septiembre (27/09).
	\2 Control 2:  Semana del 28 Octubre (01/11)
	\2 Control 3:  Semana del 25 Noviembre (29/11) .

	\uncover<2->{
	\1 Tareas (25\%, potenciales fechas de entrega):
	\2 Tarea 1: Semana del 09 Septiembre (13/09).
	\2 Tarea 2: Semana del 18 Noviembre (22/11). 
	}
	\uncover<3->{
	\2 Examen  recuperativo: 09 Diciembre.
	}
	
  \end{outline}
  \end{frame}
  
  \begin{frame}{Condiciones y Políticas de Evaluación}
  \small
  \begin{outline}
   \1 Se evaluará el aprendizaje del contenido presentado en las cátedras y en las ayudantías, mediante dos actividades complementarias (tareas, ejercicios) y tres controles de cátedra. Las ponderaciones de cada instancia de evaluación son las siguientes:
   \begin{itemize}
   \1 1. Calificaciones en actividades complementarias 25\%.
  \1  2. Calificaciones en controles de cátedra 75\%.
  \end{itemize}
   \1 La Nota Final del curso se calculará considerando las ponderaciones anteriores
    \1 La aprobación de la asignatura está sujeta a las condiciones Nota Cátedra $≥$ 4.0 y Nota de Actividades Complementarias $≥$ 4.0.
  \end{outline}
  \end{frame}
  
  \begin{frame}{Condiciones y Políticas de Evaluación}
  \small
  \begin{outline}      
   \1 Estudiantes que se ausenten a un control tendrán la oportunidad de recuperarlo con el examen. La nota del examen reemplazará la nota más baja de los controles de la asignatura, solo en caso de ser la nota de examen superior.
     \1 Un/a estudiante que cometa plagio sobtendrá un 1,0 en la evaluación y el caso será informado a Escuela de Ingeniería.
  \end{outline}
  \end{frame}

    
  \begin{frame}{Materiales}
  \begin{outline}
  \1 Repositorio GitHub -- \url{https://github.com/adigenova/uohpmd}
  \2 Slides Clases (Lunes y Viernes : 10:15 - 11:45)
  \2 Código 
   \1 Ayudante : X Y   (Miercoles 10:15 - 11:45)
   
  \uncover<2->{ \1 Ucampus -- \url{https://ucampus.uoh.cl/uoh/2023/2/COM4002}
  \2 Comunicación (Consultas, noticias, evaluaciones)
  \2 Planificación }
 \uncover<3->{  \1 Bibliografía 
  \2 S. Tanenbaum, M. Van Steen. Distributed Systems: Principles and Paradigms (2nd Edition). Prentice Hall, 2006
  \2 P. J. Sadalage, M. Fowler. NoSQL Distilled: A Brief Guide to the Emerging World of Polyglot Persistence. Addison-Wesley Professional, 2012 }
  \end{outline}
  \end{frame} 
  
  \begin{frame}{Resultados de Aprendizaje}
  	\begin{outline}
		\1 Conocer los principios fundamentales de diseño de sistemas distribuidos y paralelos (Linux)
		 \uncover<2->{ \1 Implementar comunicación entre procesos de una máquina (OpenMP, pthread)}
		 \uncover<3->{ \1 Implementar comunicación entre máquinas (MPI)}
		 \uncover<4->{ \1 Utilizar Nextflow/Hadoop para distribuir tareas computacionales en un clúster. }
		 \uncover<5->{ \1 Conocer la taxonomía de modelos de datos NoSQL, sus lenguajes de consulta y manipulación de datos con modelos NO-SQL.}
		 \uncover<6->{ \1 Construir  y manipular una base de datos distribuida.}
	\end{outline}
  \end{frame} 
    \begin{frame}{Consultas?}
    \centering 
    Consultas o comentarios?\\
    Muchas Gracias.
    
    \end{frame}

  
\end{document}