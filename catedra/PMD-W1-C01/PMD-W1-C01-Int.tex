\documentclass{beamer}
\usetheme{metropolis}           % Use metropolis theme
\usepackage{outlines}
\usepackage{listings}
\usepackage{pdfpages}
%\usepackage{minted}
%\usepackage{pygmentiz}


\title{Procesamiento Masivo de Datos}
\date{\today}
\author{Alex Di Genova}
\institute{Universidad de O'higgins}
\begin{document}
  \maketitle
  
  
  
  \begin{frame}{Outline}
  \tableofcontents
\end{frame}

\section{Bienvenida curso de PMD}
  \begin{frame}{Presentación}     
   % \begin{itemize}
\only<1>{
  Alex Di Genova
    \begin{outline}
    \1 2003--2008 Ingeniero en Bioinformática. 
    \1 2013-2017 Doctor en Sistemas Complejos.
    \1 2017-2021 Postdoctorado en algoritmos y cáncer (Francia).
    \1 2022 - Profesor Asistente UOH.
    \2 Di Genoma Lab 
    	\3 Combinamos el desarrollo de nuevos algoritmos, análisis de genomas y tecnologías ómicas de última generación para estudiar sistemas biológicos complejos.
    \end{outline}
   }
   
  \end{frame}
 
  {
\setbeamercolor{background canvas}{bg=}
 \includepdf[pages=-]{img/wengan.pdf}
}
 
  \section{Planificación curso PMD}
 \begin{frame}{Unidades}
  \only<1>{
    \begin{itemize}
    	\item 4 Unidades (14 semanas)
		\begin{itemize}
			\item Distribución y Paralelismo (4 semanas)
			\item Modelamiento de Procesamiento Distribuido (4 semanas)
			\item Modelos de Almacenamiento Escalable (3 semanas) 
			\item Bases de datos Distribuidas (3 semanas)
		\end{itemize}
	  \end{itemize}
}
 


  \end{frame}
  
  \begin{frame}{Evaluaciones}
  \begin{outline}
  	
  	\1 Controles (70\%):
	\2 Control 1:  Semana del 3 Octubre.
	\2 Control 2:  Semana del 28 Noviembre.
	\uncover<2->{
	\1 Tareas (30\%):
	\2 Tarea 1: Semana del 26 Septiembre.
	\2 Tarea 2: Semana del 7 Noviembre. 
	\2 Tarea 3: Semana del 28 Noviembre }
  \end{outline}
  \end{frame}
  
  \begin{frame}{Condiciones y Políticas de Evaluación}
  \small
  \begin{outline}
   \1 El promedio de actividades complementarias se considerará como un tercer control (control III) y tendrá una ponderación de 30\%. El promedio de controles I,II y III con sus respectivas ponderaciones corresponderán a la nota final del curso. El curso será aprobado con una nota promedio igual o superior a 4,0.
    
 \uncover<2->{   \1 Estudiantes que se ausenten a un control tendrán la oportunidad de recuperarlo durante el periodo correspondiente al final del semestre. El control recuperativo es de carácter \textbf{acumulativo}, por lo tanto, contendrá contenido de las cuatro unidades del curso. Adicionalmente, alumnos que quieran remplazar una calificación en un control o actividades complementarias, también podrán rendir el control recuperativo.}
    
   \uncover<3->{ \1 Un/a estudiante que cometa plagio sobtendrá un 1,0 en la evaluación y el caso será informado a Escuela de Ingeniería.}
\end{outline}

  \end{frame}
  
  \begin{frame}{Materiales}
  \begin{outline}
  \1 Repositorio GitHub -- \url{https://github.com/adigenova/uohpmd}
  \2 Clases
  \2 Código 
  \uncover<2->{ \1 Ucampus -- \url{https://ucampus.uoh.cl/uoh/2022/2/COM400}
  \2 Comunicación (Consultas, noticias, evaluaciones)
  \2 Planificacion }
 \uncover<3->{  \1 Bibliografía 
  \2 S. Tanenbaum, M. Van Steen. Distributed Systems: Principles and Paradigms (2nd Edition). Prentice Hall, 2006
  \2 P. J. Sadalage, M. Fowler. NoSQL Distilled: A Brief Guide to the Emerging World of Polyglot Persistence. Addison-Wesley Professional, 2012 }
  \end{outline}
  \end{frame} 
  
  \begin{frame}{Resultados de Aprendizaje}
  	\begin{outline}
		\1 Conocer los principios fundamentales de diseño de sistemas distribuidos y paralelos
		 \uncover<2->{ \1 Implementar comunicación entre máquinas (MPI, RMI)}
		 \uncover<3->{ \1 Utilizar Nextflow/Hadoop para distribuir tareas computacionales básicas. }
		 \uncover<4->{ \1 Conocer la taxonomía de modelos de datos NoSQL, sus lenguajes de consulta y construir una base de datos NO-SQL.}
		 \uncover<5->{ \1 Construir una base de datos distribuida.}
	\end{outline}
  \end{frame}
  

  
  
    \begin{frame}{Consultas?}
    \centering 
    Consultas o comentarios?\\
    Muchas Gracias.
    
    \end{frame}

  
\end{document}